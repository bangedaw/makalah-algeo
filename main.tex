\documentclass[conference]{IEEEtran}
\IEEEoverridecommandlockouts
% The preceding line is only needed to identify funding in the first footnote. If that is unneeded, please comment it out.
%Template version as of 6/27/2024

\usepackage{cite}
\usepackage{amsmath,amssymb,amsfonts}
\usepackage{algorithmic}
\usepackage{graphicx}
\usepackage{textcomp}
\usepackage{xcolor}
\usepackage{fancyhdr}
\usepackage[bahasa]{babel}
\usepackage{csquotes}
\usepackage[backend=biber,style=numeric]{biblatex}
\usepackage{tikz}
\usepackage{listings}
\usepackage{xcolor}
\usepackage{tcolorbox}
\usepackage{fancyvrb}
\usepackage[colorlinks=true, urlcolor=blue, linkcolor=blue]{hyperref}
\usepackage{float}

\addbibresource{referensi.bib}
\lstdefinestyle{myPythonStyle}{
    language=Python,
    basicstyle=\ttfamily\footnotesize, % Font teletype (seperti kode) dan ukuran kecil
    keywordstyle=\color{blue},         % Warna keyword (def, import, dll)
    stringstyle=\color{red},           % Warna string
    commentstyle=\color{green!50!black}, % Warna komentar
    numbers=left,                      % Menampilkan nomor baris di kiri
    numberstyle=\tiny\color{gray},     % Format nomor baris
    stepnumber=1,                      % Jarak antar nomor baris
    breaklines=true,                   % Membungkus baris yang terlalu panjang
    frame=single,                      % Memberi bingkai kotak
    tabsize=4,                         % Ukuran tab
    showstringspaces=false             % Menghilangkan simbol spasi di dalam string
}
\usetikzlibrary{tikzmark, arrows.meta, bending}

\renewcommand{\IEEEkeywordsname}{Kata Kunci}
\renewcommand{\abstractname}{Abstrak}
\renewcommand{\figurename}{Gambar}
\renewcommand{\tablename}{Tabel}
\renewcommand{\refname}{Referensi}

\pagestyle{fancy}
\fancyhf{}
\renewcommand{\headrulewidth}{0pt}
\renewcommand{\footrulewidth}{0pt}
\fancyfoot[L]{\footnotesize Makalah IF2123 Aljabar Linier dan Geometri Teknik Informatika ITB Semester I Tahun 2025/2026}
\def\BibTeX{{\rm B\kern-.05em{\sc i\kern-.025em b}\kern-.08em
    T\kern-.1667em\lower.7ex\hbox{E}\kern-.125emX}}
\begin{document}

\title{Simulasi Algoritma Kuantum Grover Menggunakan Ruang Vektor Kompleks dan Transformasi Matriks Unitary}

\author{\IEEEauthorblockN{Edward David Rumahorbo - 13524036}
\IEEEauthorblockA{\textit{Program Studi Teknik Informatika} \\
\textit{Sekolah Teknik Elektro dan Informatika}\\
\textit{Institut Teknologi Bandung}, Jl. Ganesha 10 Bandung 40132, Indonesia \\
13524036@mahasiswa.itb.ac.id, edwardrumahorbo1@gmail.com}
}

\maketitle
\thispagestyle{fancy} 

\begin{abstract}
    Makalah ini akan menjabarkan proses simulasi untuk Algoritma Pencarian Grover, sebuah algoritma kuantum yang memberikan percepatan kuadratik $O(\sqrt{N})$ untuk masalah pencarian tak terurut. Secara spesifik, akan dibuktikan juga bahwa Operator Oracle dan Diffuser merupakan manifestasi dari Refleksi Householder, yang secara geometris merotasi vektor keadaan dalam subruang dua dimensi. 
\end{abstract}

\begin{IEEEkeywords}
Algoritma Grover, Aljabar Linear, Ruang Vektor Kompleks, Refleksi Householder, Produk Tensor.
\end{IEEEkeywords}
\section{Pendahuluan}
This document is a model and instructions for \LaTeX.
Please observe the conference page limits. For more information about how to become an IEEE Conference author or how to write your paper, please visit   IEEE Conference Author Center website: https://conferences.ieeeauthorcenter.ieee.org/.

\subsection{Maintaining the Integrity of the Specifications}

The IEEEtran class file is used to format your paper and style the text. All margins, 
column widths, line spaces, and text fonts are prescribed; please do not 
alter them. You may note peculiarities. For example, the head margin
measures proportionately more than is customary. This measurement 
and others are deliberate, using specifications that anticipate your paper 
as one part of the entire proceedings, and not as an independent document. 
Please do not revise any of the current designations.
\section{Dasar Teori}

\subsection{Vektor dan Ruang Vektor $\mathbb{C}^n$}
Pada aljabar linear, sebuah vektor $v$ dalam ruang dimensi $n$ dapat direpresentasikan sebagai sebuah \textit{array} kolom berukuran $n \times 1$. Jika pada kasus klasik vektor berada di ruang bilangan riil $\mathbb{R}^n$, pada komputasi kuantum, vektor akan bekerja di ruang bilangan kompleks $\mathbb{C}^n$.

Sebuah vektor $v \in \mathbb{C}^n$ dapat ditulis sebagai
$$v = \begin{bmatrix} v_1 \\ v_2 \\ \vdots \\ v_n \end{bmatrix}, \quad \text{di mana } v_i \in \mathbb{C}$$
Dua operasi fundamental pada ruang vektor ini adalah
\begin{enumerate}
    \item Penjumlahan vektor ($u+v$) dengan melakukan operasi elemen per elemen
    \item Perkalian skalar $\alpha v$, di mana  $\alpha$ adalah bilangan kompleks
\end{enumerate}

Dalam notasi fisika (notasi Dirac), vektor kolom ini ditulis sebagai \textit{ket vector} $|v\rangle$. Hubungan ini penting untuk menerjemahkan literatur kuantum ke dalam operasi matriks standar.  
\begin{figure}[htbp]
    \centerline{\includegraphics[width=0.8\linewidth]{dirac_notation.jpg}}
    % https://thanosanprathifkumar.medium.com/quantum-mechanics-101-the-math-20983945a320
    \caption{Notasi Dirac}
    \label{fig:dirac_notation}
\end{figure}

\subsection{Matriks dan Transformasi Linear}
Dalam konteks ini, Matriks adalah susunan bilangan yang merepresentasikan sebuah Transformasi Linear antar ruang vektor. Jika $A$ adalah matriks berukuran $m \times n$, maka perkalian matriks dengan vektor $u$, atau bisa dinotasikan sebagai $v=Au$, akan memetakan vektor $u$ dari ruang $\mathbb{C}^n$ ke vektor $v$ di ruang $\mathbb{C}^m$.

Sifat linearitas ini didefinisikan sebagai $$A(\alpha u + \beta w) = \alpha (Au) + \beta (Aw)$$
Dalam algoritma Grover, semua gerbang kuantum (operasi gerbang logika dalam komputasi kuantum) adalah matriks persegi $N \times N$ yang mentransformasi \textit{state vector} sistem. Matriks ini bertindak sebagai operator yang merotasi atau memetakan vektor ke posisi baru dalam ruang vektor.

\subsection{Hasil Kali Dalam (Inner Product) dan Norma}
Untuk mengukur panjang suatu vektor dan sudut di antara 2 buah vektor, akan digunakan perhitungan Hermitian Inner Product. Untuk dua vektor $u$ dan $v$, hasil Hermitian Inner Product-nya didefinisikan sebagai $$\langle u, v \rangle = u^\dagger v = \begin{bmatrix} u_1^* & u_2^* & \dots & u_n^* \end{bmatrix} \begin{bmatrix} v_1 \\ v_2 \\ \vdots \\ v_n \end{bmatrix} = \sum_{i=1}^n u_i^* v_i$$ di mana $u^{\dagger}$, atau dibaca $u$ dagger, merupakan transpos konjugat dari vektor $u$. Vektor $u$ awalnya akan diubah terlebih dahulu menjadi matriks baris, kemudian setiap elemen kompleksnya di konjugat. 

Norma sebuah vektor didefinisikan sebagai akar kuadrat dari hasil inner product sebuah vektor dengan dirinya sendiri. Norma bisa dinotasikan sebagai $$\|v\| = \sqrt{\langle v, v \rangle} = \sqrt{\sum_{i=1}^n |v_i|^2}$$ Dalam konteks kuantum, setiap \textit{state vector} harus merupakan sebuah vektor satuan, yang berarti norma dari vektor tersebut hasus sama dengan 1. Ini berarti, probabilitas total dari semua kemungkinan keadaan harus selalu bernilai 1.

\subsection{Hasil Kali Luar (Outer Product) dan Proyeksi}
Selain \textit{inner product}, \textit{outer product} juga memegang peran penting dalam algoritma ini. Jika $|u\rangle$ adalah vektor kolom berukuran $n \times 1$, maka hasil \textit{outer product} dapat didefinisikan sebagai $$P = |u\rangle\langle u| = u u^\dagger$$

Hasil dari \textit{outer product} adalah sebuah matriks persegi berukuran $n \times n$. Jika $|u\rangle$ adalah sebuah vektor satuan, maka matriks $P$ adalah sebuah Matriks Proyeksi Ortogonal. Ini penting karena operasi pada algoritma Grover akan memproyeksikan \textit{state vektor} ke arah tertentu.

\subsection{Matriks Unitary dan Nilai Eigen}
Sebuah matriks persegi $A$ dikatakan unitary jika invers dari matriks tersebut adalah transpos konjugat nya, atau bisa dinotasikan sebagai $$U^{-1} = U^\dagger$$ $$U^\dagger U = I$$

Sifat penting dari sebuah matriks unitary adalah matriks ini memiliki nilai eigen dengan modulus bernilai 1, sehingga matriks ini akan mempertahankan norma vektornya. Jika sebuah vektor $v$ dikalikan dengan matrik unitary $A$, panjang vektor $v$ tidak akan berubah ($\|Uv\| = \|v\|$). Sifat ini akan menjamin bahwa selama proses komputasi kuantum terjadi, total probabilitas yang ada di dalam sistem akan selalu berjumlah 1. Dalam algoritma Grover, matriks akan direkayasa sehingga memiliki nilai eigen tertentu: 
\begin{itemize}
    \item Vektor solusi $|\omega\rangle$ adalah vektor eigen dari Matriks Oracle $U_{\omega}$ yang memiliki nilai eigen -1
    \item Vektor lain adalah vektor eigen dari Matriks Diffuser $U_{s}$ yang memiliki nilai eigen 1
\end{itemize}

\subsection{Refleksi Householder}
Dalam aljabar linear, matriks Householder digunakan untuk mencerminkan sebuah vektor terhadap sebuah bidang yang tegak lurus terhadap vektor satuan $u$. Rumus umum dari matriks ini dapat dinotasikan sebagai $$H_u = I - 2|u\rangle\langle u|$$

Matrik ini akan membalikkan tanda komponen vektor yang sejajar dengan u, dan tidak akan mengubah sama sekali komponen yang tegak lurus terhadap vektor $u$. Matriks Householder adalah matriks Unitary dan Hermitian. Pada bagian analisis matematis, akan ditunjukkan bahwa ]\textit{Oracle} dan \textit{Difusser} hanyalah bentuk khusus dari refleksi Householder. 

\subsection{Produk Tensor (Tensor Product) dan Entanglement}
untuk menggabungkan dua buah vektor yang terpisah (misalnya 2 qubit), operasi yang dilakukan adalah \textit{tensor product}. Jika $A$ adalah matriks $ m\times n$ dan $B$ adalah matriks $p \times q$, maka $A \otimes B$ akan menghasilkan sebuah matriks berukuran $mp \times nq$. Untuk $m = p = 2$ dan $n = q = 1$, hasil produk tensor $A \otimes B$ dapat dinotasikan sebagai $$\begin{bmatrix} a \\ b \end{bmatrix} \otimes \begin{bmatrix} c \\ d \end{bmatrix} = \begin{bmatrix} a \begin{bmatrix} c \\ d \end{bmatrix} \\ b \begin{bmatrix} c \\ d \end{bmatrix} \end{bmatrix} = \begin{bmatrix} ac \\ ad \\ bc \\ bd \end{bmatrix}$$ 
Untuk vektor kompleks, pertumbuhan ukuran ruang vektor bisa digambarkan seperti berikut: $$\mathcal{H} = \mathbb{C}^2 \otimes \dots \otimes \mathbb{C}^2 \cong \mathbb{C}^{2^n}$$ Operasi ini akan menyebabkan matriks di ruang vektor kuantum bertumbuh secara eksponensial seiring bertambahnya jumlah qubit ($2^n$), dan inilah yang menjadi tantangan utama dalam simulasi klasik. Keunikan yang muncul pada proses ini adalah \textit{entanglement}, di mana \textit{state vector} gabungan tidak bisa difaktorkan kembali menjadi vektor-vektor individu, atau dengan kata lain $|\psi\rangle \neq |a\rangle \otimes |b\rangle$. 

\section{Analisis Matematis Algoritma Grover}
\section{Metodologi Simulasi: Pendekatan Aljabar Linear Klasik}
\section{Implementasi dan Eksperimen}
Pengimplementasian simulasi dilakukan dalam Python untuk memvisualisasikan rotasi vektor.

\subsection{Kode Simulasi}
Berikut adalah kode Python dari \textit{function} yang digunakan untuk simulasi algoritma Grover. Beberapa library yang digunakan untuk membantu implementasi simulasi algoritma Grover adalah numpy dan matplotlib (untuk mempersingkat kode, digunakan singkatan np untuk numpy dan plt untuk matplotlib). Kode lengkap untuk simulasi ini bisa dilihat melalui laman github \url{https://github.com/bangedaw/makalah-algeo.git} atau pada bagian lampiran. 

\subsection{Hasil Simulasi}
Berikut adalah hasil simulasi yang dilakukan dengan menggunakan beberapa nilai $n$ yang berbeda 

\begin{figure}[H]
    \centerline{\includegraphics[width=0.8\linewidth]{n=4.png}}
    \caption{Hasil simulasi untuk $n=4$}
    \label{fig:n=4}
\end{figure}

\begin{figure}[H]
    \centerline{\includegraphics[width=0.8\linewidth]{n=6.png}}
    \caption{Hasil simulasi untuk $n=6$}
    \label{fig:n=6}
\end{figure}

\begin{figure}[H]
    \centerline{\includegraphics[width=0.8\linewidth]{n=8.png}}
    \caption{Hasil simulasi untuk $n=8$}
    \label{fig:n=8}
\end{figure}

\begin{figure}[H]
    \centerline{\includegraphics[width=0.8\linewidth]{n=10.png}}
    \caption{Hasil simulasi untuk $n=10$}
    \label{fig:n=10}
\end{figure}

\subsection{Analisis Hasil dan Perbandingan}
Hasil simulasi dengan menggunakan nilai $n=10$ menunjukkan beberapa perilaku yang sangat menarik, di mana
\begin{itemize}
    \item Probabilitas meningkat secara monoton hingga melewati $k=25$ hingga mendekati $|\omega\rangle$. Ini sangat dekat dengan prediksi teoritis $k \approx \frac{\pi}{4}\sqrt{1024} \approx 25.12$.
    \item Jika iterasi terus dilanjutkan hingga melewati $k=25$, maka akan terjadi penurunan probabilitas kembali menuju ke nol. Ini membuktikan bahwa algoritma Grover bukanlah proses yang konvergen asimtotik, melainkan proses rotasi siklik. Secara geometris, \textit{state vector} telah melewati vektor target $|\omega\rangle$, dan ini berimplikasi pada keharustauan penguji tentang kapan harus berhenti melakukan iterasi. 
\end{itemize}
\section{Kesimpulan}
Melalui makalah ini, algoritma Grover telah berhasil disimulasikan menggunakan komputer klasik dengna pendekatan Aljabar Linear. Kesimpulan utama dari makalah ini adalah: 
\begin{itemize}
    \item Algooritma Grover bekerja berdasarkan prinsip rotasi vektor dalam ruang hasil kali dalam kompleks, yang digerakkan oleh dua refleksi Householder.
    \item Simulasi memvalidasi bahwa jumlah langkah yang diperlukan sebanding dengan $\sqrt{N}$, dibuktikan oleh periode fungsi sinus probabilitas.
    \item Simulasi klasik menghadapi batasan memori eksponensial akibat sifat produk tensor. Selain itu, stabilitas numerik menjadi faktor kritis; simulasi jangka panjang memerlukan presisi aritmatika tinggi untuk mencegah loss of unitarity.
\end{itemize}

\section*{Lampiran}
\begin{lstlisting}[style=myPythonStyle, label={code:python1}]
    def visualize_grover_process(n_qubits, target_index):
    N = 2**n_qubits
    
    H1 = np.array([[1, 1], [1, -1]]) / np.sqrt(2)
    
    H_n = H1
    for _ in range(n_qubits - 1):
        H_n = np.kron(H_n, H1)
        
    # Inisialisasi State |0...0>
    state = np.zeros(N, dtype=np.complex128)
    state[0] = 1.0 
    
    # Buat Superposisi Awal |s>
    state = H_n @ state
    s_vec = state.copy().reshape(-1, 1)
    
    # Oracle Matrix: I - 2|w><w| (Matriks Diagonal)
    Oracle = np.eye(N, dtype=np.complex128)
    Oracle[target_index, target_index] = -1
    
    # Diffuser Matrix: 2|s><s| - I (Refleksi Householder)
    Diffuser = 2 * (s_vec @ s_vec.conj().T) - np.eye(N)
    
    w_vec = np.zeros(N)
    w_vec[target_index] = 1.0
    
    # Vektor basis |s'> (Gram-Schmidt)
    s_prime = s_vec.flatten() - (np.vdot(w_vec, s_vec.flatten()) * w_vec)
    norm_s_prime = np.linalg.norm(s_prime)
    
    if norm_s_prime > 1e-10:
        s_prime = s_prime / norm_s_prime
    
    # Fungsi koordinat
    def get_coords(vec):
        y = np.abs(np.vdot(w_vec, vec))       # Proyeksi ke Target
        x = np.abs(np.vdot(s_prime, vec))     # Proyeksi ke Non-Target
        return x, y

    optimal_iter = int(np.pi / 4 * np.sqrt(N))
    print(f"Ruang Pencarian N={N}, Target Indeks={target_index}")
    print(f"Iterasi Optimal Teoritis: {optimal_iter}")
    
    amplitudes_history = []
    coords_history = []
    
    # Simpan kondisi awal
    amplitudes_history.append(state.real.copy())
    coords_history.append(get_coords(state))
    
    for i in range(optimal_iter):
        # a. Terapkan Oracle
        state = Oracle @ state
        
        # b. Terapkan Diffuser
        state = Diffuser @ state
        
        # Simpan data
        amplitudes_history.append(state.real.copy())
        coords_history.append(get_coords(state))

    fig = plt.figure(figsize=(14, 6))
    
    # Plot A: Evolusi Amplitudo
    ax1 = fig.add_subplot(1, 2, 1)
    
    indices = np.arange(N)
    ax1.bar(indices, amplitudes_history[0], color='blue', alpha=0.3, label='Awal (|s>)')
    ax1.bar(indices, amplitudes_history[-1], color='red', alpha=0.7, label=f'Akhir (Iterasi {optimal_iter})')
    ax1.bar([target_index], [amplitudes_history[-1][target_index]], color='green')
    
    ax1.set_xlabel('Index State (Basis Komputasi)')
    ax1.set_ylabel('Amplitudo (Bagian Real)')
    ax1.set_title(f'Amplifikasi Amplitudo pada Target {target_index}')
    ax1.set_ylim(-1.1, 1.1)
    ax1.axhline(0, color='black', linewidth=0.8)
    ax1.legend()
    
    # Plot B: Rotasi Geometris
    ax2 = fig.add_subplot(1, 2, 2)
    
    xs = [c[0] for c in coords_history]
    ys = [c[1] for c in coords_history]
    
    # Plot jejak rotasi
    ax2.plot(xs, ys, 'o--', color='purple', label='Lintasan Vektor')
    
    # Gambar panah
    for i in range(len(xs)):
        ax2.quiver(0, 0, xs[i], ys[i], angles='xy', scale_units='xy', scale=1, color='purple', alpha=0.3)
        ax2.text(xs[i], ys[i] + 0.02, f'{i}', fontsize=10, fontweight='bold', color='darkblue')

    ax2.set_xlim(-0.1, 1.1)
    ax2.set_ylim(-0.1, 1.1)
    ax2.set_xlabel("Komponen |s'> (Bukan Solusi)")
    ax2.set_ylabel("Komponen |w> (Solusi)")
    ax2.set_title("Rotasi Vektor Keadaan (2D Plane)")
    ax2.grid(True)
    
    circle = plt.Circle((0, 0), 1, color='gray', fill=False, linestyle=':')
    ax2.add_artist(circle)
    
    plt.tight_layout()
    plt.show()
\end{lstlisting}


\nocite{*}
\printbibliography

\section*{Pernyataan}
Dengan ini saya menyatakan bahwa makalah yang saya tulis ini adalah tulisan saya sendiri, bukan saduran, atau terjemahan dari makalah orang lain, dan bukan plagiasi.
\vspace{1mm}
Bandung, 24 Desember 2025
\vspace{1mm}
\begin{figure}[H]
    \centerline{\includegraphics[width=0.8\linewidth]{Tanda tangan_Edward David Rumahorbo.jpg}}
\end{figure}
Edward David Rumahorbo
13524036

\end{document}
