\section{Dasar Teori}

\subsection{Vektor dan Ruang Vektor $\mathbb{C}^n$}
Pada aljabar linear, sebuah vektor $v$ dalam ruang dimensi $n$ dapat direpresentasikan sebagai sebuah \textit{array} kolom berukuran $n \times 1$. Jika pada kasus klasik vektor berada di ruang bilangan riil $\mathbb{R}^n$, pada komputasi kuantum, vektor akan bekerja di ruang bilangan kompleks $\mathbb{C}^n$.

Sebuah vektor $v \in \mathbb{C}^n$ dapat ditulis sebagai
$$v = \begin{bmatrix} v_1 \\ v_2 \\ \vdots \\ v_n \end{bmatrix}, \quad \text{di mana } v_i \in \mathbb{C}$$
Dua operasi fundamental pada ruang vektor ini adalah
\begin{enumerate}
    \item Penjumlahan vektor ($u+v$) dengan melakukan operasi elemen per elemen
    \item Perkalian skalar $\alpha v$, di mana  $\alpha$ adalah bilangan kompleks
\end{enumerate}

Dalam notasi fisika (notasi Dirac), vektor kolom ini ditulis sebagai \textit{ket vector} $|v\rangle$. Hubungan ini penting untuk menerjemahkan literatur kuantum ke dalam operasi matriks standar.  
\begin{figure}[htbp]
    \centerline{\includegraphics[width=0.8\linewidth]{dirac_notation.jpg}}
    % https://thanosanprathifkumar.medium.com/quantum-mechanics-101-the-math-20983945a320
    \caption{Notasi Dirac}
    \label{fig:dirac_notation}
\end{figure}

\subsection{Matriks dan Transformasi Linear}
Dalam konteks ini, Matriks adalah susunan bilangan yang merepresentasikan sebuah Transformasi Linear antar ruang vektor. Jika $A$ adalah matriks berukuran $m \times n$, maka perkalian matriks dengan vektor $u$, atau bisa dinotasikan sebagai $v=Au$, akan memetakan vektor $u$ dari ruang $\mathbb{C}^n$ ke vektor $v$ di ruang $\mathbb{C}^m$.

Sifat linearitas ini didefinisikan sebagai 
$$A(\alpha u + \beta w) = \alpha (Au) + \beta (Aw)$$
Dalam algoritma Grover, semua gerbang kuantum (operasi gerbang logika dalam komputasi kuantum) adalah matriks persegi $N \times N$ yang mentransformasi vektor keadaan sistem. Matriks ini bertindak sebagai operator yang merotasi atau memetakan vektor ke posisi baru dalam ruang vektor.

\subsection{Hasil Kali Dalam (Inner Product) dan Norma}


\subsection{Matriks Unitary dan Ortogonalitas}

\subsection{Produk Tensor (Tensor Product)}
