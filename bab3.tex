\section{Analisis Matematis Algoritma Grover}
Algoritma Grover bekerja dengan melakukan rotasi pada vektor awal $|s\rangle$ hingga mendekati vektor solusi $|\omega\rangle$ melalui dua operasi matriks utama.

\subsection{Menginisialisasi Superposisi Seragam}
Pada mulanya, sistem akan disiapkan dalam keadaan superposisi seragam dengan menggunakan gerbang Hadamard. Vektor ini merupakan kombinasi linear dari semua basis dengan bobot yang sama. $$|s\rangle = \frac{1}{\sqrt{N}} \sum_{x=0}^{N-1} |x\rangle$$

\subsection{Konstruksi Oracle ($U_\omega$) dan Diffuser ($U_s$)}
Oracle adalah sebuah fungsi untuk menandai target atau solusi $|\omega \rangle$. Secara matematis Oracle didefinisikan sebagai matriks yang membalik fase vektor solusi $|\omega \rangle$ dan membiarkan vektor ortogonalnya tetap. $$U_\omega |\omega\rangle = -|\omega\rangle$$ $$U_\omega |x\rangle = |x\rangle \quad \text{untuk } x \neq \omega$$ Dalam bentuk matriks, Oracle adalah refleksi Householder terhadap vektor solusi. $$U_\omega = I - 2|\omega\rangle\langle\omega|$$ Ini adalah matriks diagonal dengan entri $-1$ pada indeks solusi dan 1 di 1 di tempat lain

Diffuser adalah refleksi Householder terhadap vektor superposisi awal $|s \rangle$. Diffuser bertujuan untuk melakukan "inversi terhadap rata-rata" $$U_s = 2|s\rangle\langle s| - I = -(I - 2|s\rangle\langle s|)$$ Matriks $|s\rangle \langle s|$ adalah \textit{outer product} yang setiap elemennya bernilai $1/N$. Matriks Diffuser padat dengan setiap elemen diagonal bernilai $2/N -1$ dan elemen non-diagonal bernilai $2/N$.
\subsection{Rotasi Bidang sebagai Interpretasi Geometris}
Dengan meninjau bahwa bidang yang direntang oleh vektor solusi $|\omega\rangle$ dan vektor superposisi elemen selain solusi $|s'\rangle$, vektor superposisi awal $|s\rangle$ dapat ditulis sebagai $$|s\rangle = \sin(\theta/2) |\omega\rangle + \cos(\theta/2) |s'\rangle$$ di mana $\sin(\theta/2) = 1/\sqrt{N}$. Langkah iterasi Grover adalah $G = U_s U_\omega$. Karena kita tau bahwa $U_s$ dan $U_\omega$ adalah sebuah matriks refleksi, maka bisa dikatakan bahwa komposisi $G$ adalah sebuah rotasi. Dalam geometri bidang, dua refleksi terhadap garis yang membentuk sudut $\alpha$ akan ekuivalen dengan rotasi sebesar $2\alpha$. Pada keadaan ini, sudut antara garis ortogonal $|\omega\rangle$ dan garis $|s\rangle$ adalah sebesar $\pi/2 - \theta/2$. Akibatnya, operasi Grover akan merotasi \textit{state vektor} sebesar sudut $\theta$ mendekati $|\omega\rangle$. Dalam $k$ iterasi, vektor akan berubah menjadi $$G^k |s\rangle = \sin\left((2k+1)\frac{\theta}{2}\right) |\omega\rangle + \cos\left((2k+1)\frac{\theta}{2}\right) |s'\rangle$$ dan probabilitas sukses adalah kuadrat amplitudo komponen $|\omega\rangle$: $$P(k) = \sin^2\left((2k+1)\frac{\theta}{2}\right)$$