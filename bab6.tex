\section{Kesimpulan}
Melalui makalah ini, algoritma Grover telah berhasil disimulasikan menggunakan komputer klasik dengna pendekatan Aljabar Linear. Kesimpulan utama dari makalah ini adalah: 
\begin{itemize}
    \item Algooritma Grover bekerja berdasarkan prinsip rotasi vektor dalam ruang hasil kali dalam kompleks, yang digerakkan oleh dua refleksi Householder.
    \item Simulasi memvalidasi bahwa jumlah langkah yang diperlukan sebanding dengan $\sqrt{N}$, dibuktikan oleh periode fungsi sinus probabilitas.
    \item Simulasi klasik menghadapi batasan memori eksponensial akibat sifat produk tensor. Selain itu, stabilitas numerik menjadi faktor kritis; simulasi jangka panjang memerlukan presisi aritmatika tinggi untuk mencegah loss of unitarity.
\end{itemize}
