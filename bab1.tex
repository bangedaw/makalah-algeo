\section{Pendahuluan}
Bayangkan jika anda memiliki sebuah kumpulan bilangan acak dan sebuah fungsi yang akan mengembalikan nilai \textit{true} hanya untuk satu bilangan spesifik dari kumpulan tersebut. Tantangannya adalah anda tidak boleh melihat isi ataupun kode dari fungsi tersebut. Dengan menggunakan komputer klasik, pendekatan terbaik yang bisa anda lakukan adalah mengecek bilangan satu-persatu. Untuk $N$ elemen, anda perlu memeriksa rata-rata $N/2$ kali, sehingga kokmkpleksitas algoritma yang dihasilkan adalah $O(N)$. Untuk nilai $N$ yang sangat besar, tentunya kecepatan komputasi akan menjadi hal yang sangat krusial dan membuat pendekatan ini menjadi tidak efisien.

Jenis permasalahan seperti ini dikategorikan dalam kelas malasah di mana solusi eksaknya sulit ditemukan, tetapi kebenaran dari solusi tersebut sangat mudah divalidasi, atau sering disebut sebagai \textit{NP Problem}. Lov Grover memperkenalkan sebuah algoritma kuantum yang dapat menyelesaikan masalah pencarian pada basis data tak terurut ini dengan kompleksitas $O(\sqrt{N})$. Peningkatan kecepatan ini sangat signifikan. Sebagai contoh, untuk mencari sebuah item dalam satu juta data, komputer klasik membutuhkan rata-rata membutuhkan 500.000 langkah, sedangkan dengan Algoritma Grover, hanya dibutuhkan sekitar 1.000 langkah. 

Namun, memahami bagaimana kenaikan kecepatan ini terjadi seringkali terhalang oleh notasi fisika kuantum yang tidak umum. Padahal, jika ditelusuri lebih dalam, Algoritma Grover sebenarnya adalah aplikasi dari Aljabar Linear dan Geometri. Sistem kuantum tidak lain adalah vektor di dalam Ruang Vektor Umum (khususnya ruang vektor kompleks), dan operasi komputasi yang terjadi adalah Transformasi Linear yang dilakukan dengan matriks. Algoritma ini tidak bekerja dengan menebak secara lebih cepat, tetapi dengan memanipulasi \textit{state vector} melalui rotasi dan  pencerminan terhadap vektor tersebut agar mendekati vektor solusi yang diinginkan. 

Malakah ini bertujuan untuk membedah Algoritma Grover menggunakan pendekatan Aljabar Linear seperti matriks, vektor, \textit{inner product}, dan \textit{unitary tranformation}, serta imlementasi dari algoritma ini menggunakan bahasa pemrograman Python sebagai visualisasi mengenai bagaimana operasi matriks dapat memanipulasi probabilitas, memberikan bukti konkret atas teori yang dibahas. 

\begin{figure}[htbp]
    \centerline{\includegraphics[width=0.8\linewidth]{quantum_computing.jpg}}
    % https://x.com/LuoErik8lrl/status/1880329244871471347?s=20
    \caption{Komputer Kuantum}
    \label{fig:quantum_computing}
\end{figure}