\section{Dasar Teori}

\subsection{Vektor dan Ruang Vektor $\mathbb{C}^n$}
Pada aljabar linear, sebuah vektor $v$ dalam ruang dimensi $n$ dapat direpresentasikan sebagai sebuah \textit{array} kolom berukuran $n \times 1$. Jika pada kasus klasik vektor berada di ruang bilangan riil $\mathbb{R}^n$, pada komputasi kuantum, vektor akan bekerja di ruang bilangan kompleks $\mathbb{C}^n$.

Sebuah vektor $v \in \mathbb{C}^n$ dapat ditulis sebagai
$$v = \begin{bmatrix} v_1 \\ v_2 \\ \vdots \\ v_n \end{bmatrix}, \quad \text{di mana } v_i \in \mathbb{C}$$
Dua operasi fundamental pada ruang vektor ini adalah
\begin{enumerate}
    \item Penjumlahan vektor ($u+v$) dengan melakukan operasi elemen per elemen
    \item Perkalian skalar $\alpha v$, di mana  $\alpha$ adalah bilangan kompleks
\end{enumerate}

Dalam notasi fisika (notasi Dirac), vektor kolom ini ditulis sebagai \textit{ket vector} $|v\rangle$. Hubungan ini penting untuk menerjemahkan literatur kuantum ke dalam operasi matriks standar.  
\begin{figure}[htbp]
    \centerline{\includegraphics[width=0.8\linewidth]{dirac_notation.jpg}}
    % https://thanosanprathifkumar.medium.com/quantum-mechanics-101-the-math-20983945a320
    \caption{Notasi Dirac}
    \label{fig:dirac_notation}
\end{figure}

\subsection{Matriks dan Transformasi Linear}
Dalam konteks ini, Matriks adalah susunan bilangan yang merepresentasikan sebuah Transformasi Linear antar ruang vektor. Jika $A$ adalah matriks berukuran $m \times n$, maka perkalian matriks dengan vektor $u$, atau bisa dinotasikan sebagai $v=Au$, akan memetakan vektor $u$ dari ruang $\mathbb{C}^n$ ke vektor $v$ di ruang $\mathbb{C}^m$.

Sifat linearitas ini didefinisikan sebagai $$A(\alpha u + \beta w) = \alpha (Au) + \beta (Aw)$$
Dalam algoritma Grover, semua gerbang kuantum (operasi gerbang logika dalam komputasi kuantum) adalah matriks persegi $N \times N$ yang mentransformasi \textit{state vector} sistem. Matriks ini bertindak sebagai operator yang merotasi atau memetakan vektor ke posisi baru dalam ruang vektor.

\subsection{Hasil Kali Dalam (Inner Product) dan Norma}
Untuk mengukur panjang suatu vektor dan sudut di antara 2 buah vektor, akan digunakan perhitungan Hermitian Inner Product. Untuk dua vektor $u$ dan $v$, hasil Hermitian Inner Product-nya didefinisikan sebagai $$\langle u, v \rangle = u^\dagger v = \begin{bmatrix} u_1^* & u_2^* & \dots & u_n^* \end{bmatrix} \begin{bmatrix} v_1 \\ v_2 \\ \vdots \\ v_n \end{bmatrix} = \sum_{i=1}^n u_i^* v_i$$ di mana $u^{\dagger}$, atau dibaca $u$ dagger, merupakan transpos konjugat dari vektor $u$. Vektor $u$ awalnya akan diubah terlebih dahulu menjadi matriks baris, kemudian setiap elemen kompleksnya di konjugat. 

Norma sebuah vektor didefinisikan sebagai akar kuadrat dari hasil inner product sebuah vektor dengan dirinya sendiri. Norma bisa dinotasikan sebagai $$\|v\| = \sqrt{\langle v, v \rangle} = \sqrt{\sum_{i=1}^n |v_i|^2}$$ Dalam konteks kuantum, setiap \textit{state vector} harus merupakan sebuah vektor satuan, yang berarti norma dari vektor tersebut hasus sama dengan 1. Ini berarti, probabilitas total dari semua kemungkinan keadaan harus selalu bernilai 1.

\subsection{Hasil Kali Luar (Outer Product) dan Proyeksi}
Selain \textit{inner product}, \textit{outer product} juga memegang peran penting dalam algoritma ini. Jika $|u\rangle$ adalah vektor kolom berukuran $n \times 1$, maka hasil \textit{outer product} dapat didefinisikan sebagai $$P = |u\rangle\langle u| = u u^\dagger$$

Hasil dari \textit{outer product} adalah sebuah matriks persegi berukuran $n \times n$. Jika $|u\rangle$ adalah sebuah vektor satuan, maka matriks $P$ adalah sebuah Matriks Proyeksi Ortogonal. Ini penting karena operasi pada algoritma Grover akan memproyeksikan \textit{state vektor} ke arah tertentu.

\subsection{Matriks Unitary dan Nilai Eigen}
Sebuah matriks persegi $A$ dikatakan unitary jika invers dari matriks tersebut adalah transpos konjugat nya, atau bisa dinotasikan sebagai $$U^{-1} = U^\dagger$$ $$U^\dagger U = I$$

Sifat penting dari sebuah matriks unitary adalah matriks ini memiliki nilai eigen dengan modulus bernilai 1, sehingga matriks ini akan mempertahankan norma vektornya. Jika sebuah vektor $v$ dikalikan dengan matrik unitary $A$, panjang vektor $v$ tidak akan berubah ($\|Uv\| = \|v\|$). Sifat ini akan menjamin bahwa selama proses komputasi kuantum terjadi, total probabilitas yang ada di dalam sistem akan selalu berjumlah 1. Dalam algoritma Grover, matriks akan direkayasa sehingga memiliki nilai eigen tertentu: 
\begin{itemize}
    \item Vektor solusi $|\omega\rangle$ adalah vektor eigen dari Matriks Oracle $U_{\omega}$ yang memiliki nilai eigen -1
    \item Vektor lain adalah vektor eigen dari Matriks Diffuser $U_{s}$ yang memiliki nilai eigen 1
\end{itemize}

\subsection{Refleksi Householder}
Dalam aljabar linear, matriks Householder digunakan untuk mencerminkan sebuah vektor terhadap sebuah bidang yang tegak lurus terhadap vektor satuan $u$. Rumus umum dari matriks ini dapat dinotasikan sebagai $$H_u = I - 2|u\rangle\langle u|$$

Matrik ini akan membalikkan tanda komponen vektor yang sejajar dengan u, dan tidak akan mengubah sama sekali komponen yang tegak lurus terhadap vektor $u$. Matriks Householder adalah matriks Unitary dan Hermitian. Pada bagian analisis matematis, akan ditunjukkan bahwa ]\textit{Oracle} dan \textit{Difusser} hanyalah bentuk khusus dari refleksi Householder. 

\subsection{Produk Tensor (Tensor Product) dan Entanglement}
untuk menggabungkan dua buah vektor yang terpisah (misalnya 2 qubit), operasi yang dilakukan adalah \textit{tensor product}. Jika $A$ adalah matriks $ m\times n$ dan $B$ adalah matriks $p \times q$, maka $A \otimes B$ akan menghasilkan sebuah matriks berukuran $mp \times nq$. Untuk $m = p = 2$ dan $n = q = 1$, hasil produk tensor $A \otimes B$ dapat dinotasikan sebagai $$\begin{bmatrix} a \\ b \end{bmatrix} \otimes \begin{bmatrix} c \\ d \end{bmatrix} = \begin{bmatrix} a \begin{bmatrix} c \\ d \end{bmatrix} \\ b \begin{bmatrix} c \\ d \end{bmatrix} \end{bmatrix} = \begin{bmatrix} ac \\ ad \\ bc \\ bd \end{bmatrix}$$ 
Untuk vektor kompleks, pertumbuhan ukuran ruang vektor bisa digambarkan seperti berikut: $$\mathcal{H} = \mathbb{C}^2 \otimes \dots \otimes \mathbb{C}^2 \cong \mathbb{C}^{2^n}$$ Operasi ini akan menyebabkan matriks di ruang vektor kuantum bertumbuh secara eksponensial seiring bertambahnya jumlah qubit ($2^n$), dan inilah yang menjadi tantangan utama dalam simulasi klasik. Keunikan yang muncul pada proses ini adalah \textit{entanglement}, di mana \textit{state vector} gabungan tidak bisa difaktorkan kembali menjadi vektor-vektor individu, atau dengan kata lain $|\psi\rangle \neq |a\rangle \otimes |b\rangle$. 
